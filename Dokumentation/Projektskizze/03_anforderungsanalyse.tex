\chapter{Anforderungen}
\label{sec:Anforderungen}

Die Anforderungen an das Honeypot-System werden in Muss-, Kann- und Soll-Kriterien unterteilt.


\section{Muss-Kriterien}
\label{subsec:Muss-Kriterien}
\begin{itemize}
\item Honeypot simuliert eine Auswahl an Diensten (http, ssh, ftp)
\item Honeypot simuliert offenes WLAN-Netz / Fake-Access-Point
\item Angreifer hat keine Möglichkeit zur Interaktion mit dem Host-Betriebssystem
\item Angreifer bekommt keine bzw. nur gefälschte Antworten auf seine Anfragen
\item Überwachung des ein- und ausgehenden Netzwerkverkehrs
\item Protokollierung darf für Angreifer nicht sichtbar sein
\item Protokollierte Daten dürfen durch Angreifer nicht verändert werden können
\item Honeypot muss jederzeit deaktivierbar sein
\item Honeypot muss für Zeit X online / aktiv sein, um ein realistisches Angriffsziel zu sein
\item Honeypot ist im Internet erreichbar
\item Produktivsysteme dürfen keiner Gefahr ausgesetzt werden
\item Honeypot muss in DMZ sein und darf keine Gefahr für unbeteiligte Dritte darstellen
\end{itemize}

\newpage

\section{Soll-Kriterien}
\label{subsec:Soll-Kriterien}
\begin{itemize}
\item automatisierte Auswertung von Logdaten
\item automatische Benachrichtigung, wenn System angegriffen wird
\item Honeypot kann einfach zurückgesetzt / neu aufgesetzt werden (beispielsweise durch Skript zur automatischen Einrichtung / Konfiguration)
\item Logdateien / ausgewertete Dateien werden automatisch separat gespeichert (extra System, Cloud-Speicher)
\end{itemize}


\section{Kann-Kriterien}
\label{subsec:Kann-Kriterien}
\begin{itemize}
\item geringer Stromverbrauch von Honeypot-System
\item kostengünstiger Versuchsaufbau
\item Reverse DNS-Lookup von Angreifer-IP-Adresse(n)
\item Simulation weiterer Geräte (Router, Firewall, PC)
\end{itemize}


