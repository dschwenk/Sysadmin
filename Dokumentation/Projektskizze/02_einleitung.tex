\pagenumbering{arabic}
\chapter{Einleitung}
\label{sec:einleitung}

Das Internet und die Digitalisierung, die in alle Lebensbereiche Einzug erhält, verändern Gesellschaft, Wirtschaft und Kultur.
Egal ob im privaten oder beruflichen Umfeld, ständig sind wir von Computern in Form von Arbeitsgeräten, Smartphones oder anderen Geräten umgeben.
Diese Vernetzung wird in den nächsten Jahren im Zuge der "`Internet-der-Dinge-Evolution"' weiter drastisch zunehmen.\\



Ein oft vernachlässigter Aspekt hierbei ist das Thema "`IT-Sicherheit"'. Keine Software ist frei von Fehlern und Sicherheitslücken. Es bedarf einen großen Aufwand, um eine Infrastruktur vor möglichen Angriffen zu schützen.


\section{Motivation}
\label{subsec:Motivation}
Durch unsere privaten wie auch beruflichen Tätigkeiten im Systemadministratorenumfeld werden auch wir mit dem Thema der Absicherung von Infrastrukturen konfrontiert.\\


sind beide als Systemadministratoren tätig betreuen eigene Netzwerkumgebungen, wollen wissen über mögliche Angriffe sammeln, wollen kleines kostengünstiges, einfach zu kofigruerendes honeypot system haben, wollen erfahrungen sammeln, ...?


\section{Ziele}
\label{subsec:Ziele}
Ziel dieser Arbeit ist es, ein System zu entwickeln, dass als Honeypot dient. Dieser Honeypot soll eingesetzt werden, um Informationen über Angriffsmuster und Angreiferverhalten zu erhalten. Dazu stellt das System ein vermeintlich leicht angreifbares Ziel dar.\\

Jegliche Zugriffe und Aktivitäten die ein Angriff hinterlässt werden protokolliert und ausgewertet. 
Mit Hilfe von diesem Wissen kann eine reale Netzwerkumgebung gegen Angriffe abgesichert werden.\\

\begin{itemize}
\item Primärziel: einsatzfähige(r) Honeypot(s) - sicher, authentisch und lehrreich
\item Sekundärziel: von Angreifern lernen
\end{itemize}

\section{Eigene Leistung}
\label{subsec:Eigene Leistung}

Die Bereitstellung einer für einen potentiellen Angreifer authentische Umgebung, sowie die Einbettung des Honeypots in selbige.

\begin{itemize}
\item Bereitstellung einer authentischen Umgebung für den Honeypot
\item Gewährleistung der Sicherheit für das eigene und das globale Netz
\item Inbetriebnahme des Honeypots
\item Auswertung von Log-Files et cetera
\end{itemize}


