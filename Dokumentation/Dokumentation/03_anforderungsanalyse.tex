\chapter{Anforderungen}
\label{ch:Anforderungen}

Die Anforderungen an das Honeypot-System werden in Muss-, Kann- und Soll-Kriterien unterteilt.


\section{Muss-Kriterien}
\label{sec:Muss-Kriterien}
\begin{itemize}
\item Honeypot ist über das Internet erreichbar
\item Honeypot darf keinerlei Gefahr für andere Systeme darstellen
\item Honeypot muss jederzeit deaktivierbar sein
\item Angreifer darf keinerlei Möglichkeit zur Interaktion mit dem Host-Betriebssystem haben
\item Angreifer darf keine bzw. nur gefälschte Antworten auf Anfragen erhalten
\item Honeypot muss mindestens einen, besser jedoch mehrere Dienste, wie beispielsweise HTTP, SSH oder FTP, simulieren/anbieten
\item Honeypot muss ein realistisch wirkendes Angriffsziel darstellen
\item Angriffe werden geloggt
\item ein- und ausgehender Netzwerkverkehr muss (überwacht und) geloggt werden 
\item protokollierte Daten dürfen durch Angreifer nicht verändert werden können

\end{itemize}

\newpage

\section{Soll-Kriterien}
\label{sec:Soll-Kriterien}
\begin{itemize}
\item automatische Benachrichtigung, wenn System angegriffen wird
\item Protokollierung und ggf. Forwarding von Log-Files dürfen für den Angreifer nicht sichtbar sein
\item automatisierte Auswertung von Logdaten
\item Logdateien / ausgewertete Daten werden automatisch separat gespeichert (extra System, Cloud-Speicher)
\end{itemize}


\section{Kann-Kriterien}
\label{sec:Kann-Kriterien}
\begin{itemize}
\item geringer Stromverbrauch von Honeypot-System
\item kostengünstiger Versuchsaufbau
\item Reverse DNS-Lookup von Angreifer-IP-Adresse(n)
\item Simulation weiterer Geräte (Router, Firewall, PC)
\item Honeypot simuliert offenes WLAN-Netz / Fake-Access-Point
\end{itemize}


