\chapter{Anforderungen}
\label{ch:Anforderungen}

Die Anforderungen an das Honeypot-System werden in Muss-, Kann- und Soll-Kriterien unterteilt.


\section{Muss-Kriterien}
\label{sec:Muss-Kriterien}
\begin{itemize}
\item Honeypot ist über das Internet erreichbar
\item Die Gefahr einer Übernahme und folglich eines Missbrauchs des Honeypots muss minimal gehalten werden \textcolor{red}{(Wie erreichen wir das? Bzw. sollten wir den Punkt vll entfernen? Haben wir derzeit eine Möglichkeit festzustellen, dass die Kiste übernommen wurde? Ausgehenden Traffic überwachen?)}
\item Honeypot muss jederzeit deaktivierbar sein \textcolor{red}{(Ist das über die Web-Oberfläche jederzeit möglich, beispielsweise auch nach Warnung durch Email an uns?)}
\item Angreifer darf keinerlei Möglichkeit zur Interaktion mit dem Host-Betriebssystem haben
\item Angreifer darf keine bzw. nur gefälschte Antworten auf Anfragen erhalten \textcolor{red}{(Ist das bei Kippo so? Nein oder?)}
\item Honeypot muss mindestens einen, besser jedoch mehrere Dienste, wie beispielsweise HTTP, SSH oder FTP, simulieren/anbieten
\item Honeypot muss ein realistisch wirkendes Angriffsziel darstellen
\item Angriffe werden geloggt
\item Ein- und ausgehender Netzwerkverkehr muss (überwacht und) geloggt werden \textcolor{red}{(Direkte Umsetzung von Punkt2?)}
\item Protokollierte Daten dürfen durch Angreifer nicht verändert werden können \textcolor{red}{(Wie überprüfen wir das?)}

\end{itemize}

\newpage

\section{Soll-Kriterien}
\label{sec:Soll-Kriterien}
\begin{itemize}
\item Automatische Benachrichtigung, wenn System angegriffen wird \textcolor{red}{(Macht das derzeit überhaupt Sinn? Realisierbar? Wie?)}
\item Protokollierung und ggf. Forwarding von Log-Files dürfen für den Angreifer nicht sichtbar sein \textcolor{red}{(Bei Kippo nur durch Übernahme machbar oder? Eigenversuch auf Kippo verbinden)}
\item Automatisierte Auswertung von Logdaten
\item Logdateien / ausgewertete Daten werden automatisch separat gespeichert (extra System, Cloud-Speicher)
\end{itemize}


\section{Kann-Kriterien}
\label{sec:Kann-Kriterien}
\begin{itemize}
\item Geringer Stromverbrauch von Honeypot-System
\item Kostengünstiger Versuchsaufbau
\item Reverse DNS-Lookup von Angreifer-IP-Adresse(n)
\item Simulation weiterer Geräte (Router, Firewall, PC)
\item Honeypot simuliert offenes WLAN-Netz / Fake-Access-Point
\end{itemize}


