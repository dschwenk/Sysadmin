\chapter{Marktanalyse}
\label{ch:Marktanalyse}

Eine Marktanalyse zeigt, dass eine Vielzahl an verschiedenen Honeypot-Paketen, Skripten und Konfigurationen mit sehr unterschiedlichen Eigenschaften verfügbar sind. Darunter befinden sich sowohl kommerzielle als auch freie Lösungen.

\section{HoneyDrive}
\label{sec:HoneyDrive}
Mit HoneyDrive existiert eine Honeypot-Linux-Distribution auf Basis von Xubuntu Desktop 12.04.04 LTS. Diese Linux-Distribution bringt 10 vorinstallierte und vorkonfigurierte Honeypot-Pakete wie Kippo SSH Honeypot, Glastopf Web Honeypot oder Amun Malware Honeypot mit. Eine ausführliche Auflistung ist unter \cite{honeydrive16} gegeben. Die Distribution wird als OVA-Datei angeboten und kann so unter einer Virtualisierungssoftware ausgeführt werden. Neben den vorinstallierten Honeypot-Paketen sind des weiteren unter anderem ein Web- und Datenbankserver sowie wie PHPMyAdmin vorinstalliert.
Diese Linux-Distribution ermöglicht so einfach und schnell ein Honeypot-System aufzusetzen.\\

Das große Manko ist hier der veraltete Software-Stand. Die letzte Aktualisierung fand im Jahre 2014 statt. Dies, sowie der Overhead an vorinstallierten und vorkonfigurierten Diensten, birgt die Gefahr, dass ein potenzieller Angreifer über eine Lücke das Hostsystem kompromittieren oder übernehmen kann.

\section{Kippo}
\label{sec:Kippo}
Kippo ist ein SSH-Honeypot, entworfen um Bruteforce-Attacken sowie die komplette Interaktion des Angreifers mit der Shell zu protokollieren. Konnte sich ein Angreifer durch Eingabe der vorbestimmten Kombination aus Benutzer und Passwort einloggen, wird ihm von Kippo ein virtuelles System offengelegt. In diesem System kann der Angreifer wie gewohnt agieren \cite{Kippo16}.\\
Um der Anforderung der automatischen Benachrichtigung des Projektteams bei einem Angriff gerecht zu werden, gilt es Logdateien automatisiert zu analysieren und auszuwerten. Wird ein Bruteforce-Angriff erkannt, wird das Projektteam via Email benachrichtigt. Diese Anforderung kann Kippo ohne Anpassungen nicht leisten.\\

Ein wesentlicher Nachteil von Kippo ist die Tatsache, dass sich Tools zu seiner Erkennung im Umlauf befinden. Entsprechend versierte Angreifer werden Kippo daher frühzeitig erkennen.

\section{Honeyd}
\label{sec:Honeyd}
Honeyd wird von unix-artigen Betriebssystemen unterstützt. Er ist ein Daemon, der virtuelle Hosts in einem Netzwerk erzeugt. Diese Hosts können das Vorhandensein spezieller Betriebssysteme und Services simulieren, indem sie mit authentischen Antwortpaketen auf etwaige Anfragen, insbesondere Fingerprint-Pakete reagieren. Honeyd eignet sich besonders für die Ablenkung eines Angreifers und die Verschleierung der wirklichen Infrastruktur. Ebenso  dient er als Warnsystem, da jeder Zugriffsversuch auf einem der durch Honeyd erzeugten Hosts ein Hinweis auf unerwünschte Aktivitäten innerhalb der Infrastruktur darstellt \cite{Honeyd16}.\\

Honeyd eignet sich nicht ohne Weiteres für die Aufzeichnung komplexerer Angriffe, insbesondere solcher, die auf dem System selbst stattfinden, bietet jedoch die Möglichkeit weitere Geräte zu simulieren und somit zur Authentizität der Infrastruktur des Projektteams beizutragen.

\section{Glastopf}
\label{sec:Glastopf}

Glastopf ist ein als Webserver getarnter Honeypot. Dieses System nutzt den Umstand, dass viele Angreifer unter Zuhilfenahme von Suchmaschinen nach Schwachstellen auf Webservern suchen, indem es sich selbst bei den gängigsten Vertretern registriert. Dabei wird Fläche für gängige, webbasierte Angriffe wie SQL-Injections, Remote-Code-Execution, File-Inclusion et cetera, geboten. Von einem Angreifer eingeschleuster Code, wird in einer Sandbox ausgeführt. Alle Verbindungen und Angriffsversuche werden geloggt und in einer Datenbank protokolliert \cite{Glastopf16}.\\

Durch die Mithilfe von Webcrawlern ist es mit Glastopf möglich für die eigenen Schwachstellen Reklame zu betreiben und somit in kürzerer Zeit eine größere Menge an Angriffen auf das System zu lenken. Die optische Aufmachung der Glastopf-Startseite trägt allerdings dazu bei, dass Angriffe in sehr hohem Anteil nur automatisiert und kaum oder gar nicht in individuell gezielter Form stattfinden werden.