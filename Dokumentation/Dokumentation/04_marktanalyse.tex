\chapter{Marktanalyse}
\label{ch:Marktanalyse}

Eine Marktanalyse zeigt, dass eine Vielzahl an verschiedenen Honeypot-Paketen, Skripten und Konfigurationen mit sehr unterschiedlichen Eigenschaften verfügbar sind. Darunter befinden sich sowohl kommerzielle als auch freie Lösungen.

\section{HoneyDrive}
\label{sec:HoneyDrive}
Mit HoneyDrive existiert eine Honeypot-Linux-Distribution auf Basis von Xubuntu Desktop 12.04.04 LTS. Diese Linux-Distribution bringt 10 vorinstallierte und vorkonfigurierte Honeypot-Pakete wie Kippo SSH Honeypot, Glastopf Web Honeypot oder Amun Malware Honeypot mit. Eine ausführliche Auflistung ist unter \cite{honeydrive16} gegeben. Die Distribution wird als OVA-Datei angeboten und kann so unter einer Virtualisierungssoftware ausgeführt werden. Neben den vorinstallierten Honeypot-Paketen sind des Weiteren unter anderem ein Web- und Datenbankserver sowie wie PHPMyAdmin vorinstalliert.
Diese Linux-Distribution ermöglicht so einfach und schnell ein Honeypot-System aufzusetzen.\\


\section{Kippo}
\label{sec:Kippo}
Kippo ist ein SSH-Honeypot, entworfen um Bruteforce-Attacken sowie die komplette Interaktion des Angreifers mit der Shell zu protokollieren. Konnte sich ein Angreifer durch Eingabe der vorbestimmten Kombination aus Benutzer und Passwort einloggen, wird ihm von Kippo ein virtuelles System offengelegt. In diesem System kann der Angreifer wie gewohnt agieren \cite{Kippo16}. Die Protokollierung selbst erfolgt standardmäßig in Textdateien, alternativ kann auch eine Datenbank verwendet werden.\\



\section{Honeyd}
\label{sec:Honeyd}
Honeyd wird von Unix-artigen Betriebssystemen unterstützt. Er ist ein Daemon, der virtuelle Hosts in einem Netzwerk erzeugt. Diese Hosts können das Vorhandensein spezieller Betriebssysteme und Services simulieren, indem sie mit authentischen Antwortpaketen auf etwaige Anfragen, insbesondere Fingerprint-Pakete reagieren. Honeyd eignet sich besonders für die Ablenkung eines Angreifers und die Verschleierung der wirklichen Infrastruktur. Ebenso  dient er als Warnsystem, da jeder Zugriffsversuch auf einem der durch Honeyd erzeugten Hosts ein Hinweis auf unerwünschte Aktivitäten innerhalb der Infrastruktur darstellt \cite{Honeyd16}.\\



\section{Glastopf}
\label{sec:Glastopf}

Glastopf ist ein als Webserver getarnter Honeypot. Dieses System nutzt den Umstand, dass viele Angreifer unter Zuhilfenahme von Suchmaschinen nach Schwachstellen auf Webservern suchen, indem es sich selbst bei den gängigsten Vertretern registriert. Dabei wird Fläche für gängige, webbasierte Angriffe wie SQL-Injections, Remote-Code-Execution, File-Inclusion et cetera, geboten. Von einem Angreifer eingeschleuster Code, wird in einer Sandbox ausgeführt. Alle Verbindungen und Angriffsversuche werden geloggt und in einer Datenbank protokolliert \cite{Glastopf16}. Durch die Mithilfe von Webcrawlern ist es mit Glastopf möglich für die eigenen Schwachstellen Reklame zu betreiben und somit in kürzerer Zeit eine größere Menge an Angriffen auf das System zu lenken. 

\section{SNARE}
\label{sec:SNARE}

SNARE ist ein in Python geschriebener Webserver, der über einen HTTP-Requesthandler eine vom Administrator bereitgestellte Ordner-Struktur auf HTTP-Requests mappt. Darüber hinaus stellt dieser Honeypot Skripte bereit um bestehende Webpräsenzen zu klonen und als Web-Honeypot zu starten. So ist potentiell möglich jede beliebige bestehende Webseite, aber auch Eigenkreationen, als Web-Honeypot zu starten. Etwaige Eingaben, Nutzer-Interaktionen oder versuchte Zugriffe auf für Unbefugte nicht vorgesehene Administrationsebenen werden von SNARE geloggt \cite{Snare16}. Ein entscheidender Nachteil von SNARE ist, dass keinerlei Möglichkeit zum Logging von Source-IP-Adressen gegeben ist und eine entsprechende Erweiterung zur Aufzeichnung selbiger durch das Projektteam vorgenommen werden müsste.\\
