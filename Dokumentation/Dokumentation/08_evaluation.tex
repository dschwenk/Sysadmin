\chapter{Evaluation}
\label{ch:Evaluation}

\section{Umsetzung der Anforderungen}
\label{sec:Umsetzung der Anforderungen}

Die unter \ref{sec:Muss-Kriterien} aufgeführten Muss-Kriterien wurden vollständig erfüllt. In diesem Projekt wurde ein Honeypotsystem auf einem von providerdienste.de bereitgestellten vServer umgesetzt. Dieser vServer ist über eine öffentliche IPv4-Adresse im Internet erreichbar und implementiert einen SSH- und Webhoneypot. Durch die von providerdienste.de bereitgestellte Konsole ist es möglich, das System selbst dann, wenn keine Verbindung via SSH möglich ist oder der Server kompromittiert oder gar übernommen wurde, zu konfigurieren und darauf zuzugreifen. Es ist dem Projektteam somit jederzeit möglich, das Honeypotsystem abzuschalten, sofern die Situation es erfordern sollte. Durch die Aktualisierung aller Pakete und der Limitierung der SSH-Authentifzierung auf ein Public-Key-Verfahren wurde das Honeypotsystem bestmöglich abgesichert. Jeglicher Datenverkehr wird geloggt. Einem Angreifer ist es somit ohne den Besitz eines der Private-Keys der Projektmitglieder, Kenntnisse über aktuelle Schwachstellen bei den eingesetzten Systemen oder Infiltration des Rechenzentrums von providerdienste.de, nicht ohne Weiteres möglich sich Zugriff zu dem eingesetzten vServer zu verschaffen. Die Möglichkeit zur Manipulation von protokollierten Daten ist somit, gemessen an den zeitlichen Ressourcen des Projektteams, kleinstmöglich gehalten.\\

Das Soll-Kriterium der automatischen Benachrichtigung bei einem Angriff auf das Honeypotsystem stellte sich im Nachhinein als nicht sonderlich hilfreich dar. Speziell auf Port 22 und somit auf den SSH-Honeypot wurden täglich viele Verbindungen aufgebaut. Die Benachrichtigung darüber war für die Projektmitglieder aber wenig hilfreich, da jegliche Kommunikation und Angriffsversuche geloggt und automatisiert ausgewertet wurden. Zudem werden die auswerteten Daten und Konfigurationen automatisch über eine verschlüsselte Verbindung auf dem Cloud-Speicher Google Drive abgelegt. Die Benachrichtigung bietet so nur einen geringen Mehrwert. Bei der Archivierung der auswerteten Daten sowie dem Backup von Konfigurationsdateien ist anzumerken, dass hier ein separater Server oder ein anderes Backupmedium möglicherweise besser geeignet wäre, da es mit den Richtlinien von Google zu einem Datenschutzkonflikt kommen kann.\\

Der Soll-Anforderung einer automatisierten Auswertung von Logdaten wurde durch den Einsatz selbstgeschriebener Bash-Skripte, sowie der fertigen Lösung Pipal zur statistischen Analyse von Passwörtern entsprochen. Ebenso der automatisiert stattfindenden externen Sicherung sowohl von Log-Files selbst als auch den Log-Files zugehörige, bereits ausgewertete Daten wurde entsprochen, indem diese unter Verwendung eines Bash-Skripts auf Google-Drive hochgeladen. 

Die Kann-Kriterien unter \ref{sec:Kann-Kriterien} wurden nur teilweise erfolgreich umgesetzt. So war es dem Projektteam aus technischen Gründen nicht möglich, neben dem Honeypotsystem weitere Systeme wie einen Router, eine Firewall oder PCs in die Infrastruktur zu integrieren. Auch die Umsetzung eines Honeypots, der ein offenes WLAN oder einen Fake-Acces-Point bietet konnte nicht umgesetzt werden. Hier bietet sich Verbesserungspotential, da die durch ein Honeypot gewonnen Informationen speziell in einer größeren Umgebung wertvoll sind. So ist denkbar, aus den gesammelten IP-Adressen Firwallregeln zu erstellen, die auf einer der Infrastruktur vorgelagerten Firewall zur Wirkung kommen.
Die komplette Umsetzung des Projekts fand auf einem gemieteten vServer statt. Die Kosten liegen hier mit 9,00 Euro je Monat in einem überschaubaren Rahmen. Die Anforderung nach einem geringen Stromverbrauch kann nicht exakt beurteilt werden, da über den Stromverbrauch keine Informationen vorliegen. Auf Grund dessen, dass es sich bei unserem System um ein virtuelles System handelt, welches sich ein physikalisches System mit anderen virtuellen Systemen teilt, kann jedoch von einer gewissen Energieeffizienz ausgegangen werden. Die Anforderung an einen reverse DNS-Lookup wurde wie unter x.y beschrieben umgesetzt und zusätzlich durch eine Darstellung einer Geo-IP-Karte (siehe Anhang x.y) ergänzt

\section{Schlussfolgerungen SSH-Honeypot}
\label{sec:Schlussfolgerungen SSH-Honeypot}

Wie der Statistik im Anhang unter \nameref{app:Pipal Passwortstatistik} entnommen werden kann, wurden am SSH-Honeypot-Port ingesamt 420 Passwörter eingegeben, 130 davon sind einzigartig. Die 10 meistgenutzten Passwörter machen 35,46\% aller Passworteingaben aus. Betrachtet man die 10 meistgenutzten Passwörter genauer, fällt auf, das hier vorrangig die Zeichenketten \textit{admin} und \textit{password} sowie Zahlenfolgen wie \textit{123456} vertreten sind. Die Länge der eingegeben Passwörter betrug in über 25\% der Fälle sechs Zeichen, in 15,71\% acht Zeichen und in 14,29\% vier Zeichen. Über 80\% der eingegeben Passwörter bestanden aus acht oder weniger Zeichen. Ähnlich klare Verteilungen sind bei der Zusammensetzung von Passwörter aus Zeichenklassen zu erkennen. So bestehen über 38\% der Passwörter rein aus Buchstaben, weitere 30,48\% rein aus Zahlen.\\

Eine ähnliche einseitige Verteilung lässt sich bei den verwendeten Benutzernamen erkennen. In über 50\% der Fälle wurde der Benutzername \textit{root} verwendet. Werden die Benutzernamen \textit{admin}, \textit{user} und \textit{test} hinzugezählt, decken diese Benutzernamen über 75\% ab.\\

Aus diesen Zahlen wird vor allem eines deutlich. Die Verwendung eines der oben genannten Passwörter in Verbindung mit einem der genannten Benutzernamen birgt ein erhebliches Sicherheitsrisiko. Werden die Zugriffszeiten der Angreifer auf den SSH-Honeypot berücksichtigt, kann bei der Verwendung dieser Benutzer-Passwort-Kombinationen von einer Kompromittierung innerhalb weniger Stunden oder gar Minuten ausgegangen werden. Es sollte so zwingend sichergestellt werden, dass ein sicheres Passwort verwendet wird. Besser ist die Authentifizierung rein auf das Public-Key-Verfahren zu beschränken.\\

Es muss angenommen werden, dass Angreifer nicht nur bei SSH-Zugriffen auf einfache Kombination aus Benutzernamen und Passwörter setzen. Jedoch ist bei anderen Diensten zur Passwortauthentifizierung oftmals keine Alternative vorgesehen. In diesem Fall sollte zwingend ein sicheres Passwort gewählt werden. Um die Definition eines sicheren Passworts zu veranschaulichen, leitet das Projektteam aus der Passwortstatistik eine Passwortrichtlinie ab. Diese sieht vor, dass ein sicheres Passwort folgende Eigenschaften besitzt:

\begin{itemize}
\item Passwortlänge von mindestens 10 oder mehr Zeichen
\item Kombination aus den Zeichenklassen Klein- und Großbuchstaben, Zahlen und Sonderzeichen
\item Benutzung von Wörtern und persönlichen Daten wie Namen und Geburtsdatum vermeiden
\end{itemize}

Da selbst im Bewusstsein dieser Richtlinien dazu tendiert wird, nicht sichere Passwörter zu genieren, empfiehlt das Projektteam die Verwendung eines Passwortmanagers wie KeePass\footnote{ \textit{KeePass}: \url{http://keepass.info/}}. KeePass ist für alle gängigen Betriebssysteme verfügbar und ermöglicht es rein zufällige Passwörter zu generieren, zu verwalten und sicher zu speichern. Somit kann auch der Empfehlung für jeden Dienst ein separates Passwort zu verwenden einfach nachgekommen werden.

\section{Schlussfolgerungen Web-Honeypot}
\label{sec:Schlussfolgerungen Web-Honeypot}

Innerhalb eines Zeitraumes von einer Woche erfolgten auf dem Web-Honeypot 149 Zugriffe über HTTP und 5 POST-Requests. Angeforderte Pfade und POST-Requests können im Anhang unter \nameref{app:Auswertung snare-path-requests} und \nameref{app:Auswertung snare-post-requests} eingesehen werden. Während dieses Zeitraums bestand der Webauftritt des Web-Honeypots aus einer Kopie der Seite example.com\footnote{ \textit{example.com}: \url{http://example.com/}}. Ungeachtet der Tatsache, dass damit einem Angreifer lediglich die Möglichkeit gegeben war über verschiedene Pfade einen Zugriff zu potentiell über die Startseite nicht erreichbaren Login Seiten zu erhalten oder über Anwendung von HTTP-Methoden wie POST Schadcode einschleusen zu können und somit die angebotene Angriffsfläche sehr gering ausfiel, konnte das Projektteam mehrere verschiedene Angriffsszenarien aufzeichnen.\\

Die im folgenden Abschnitt getroffenen Aussagen basieren auf Vermutungen, die durch den aktuellen Kenntnisstand des Projektteams begründet und entsprechend zu bewerten sind.\\

61 Aufrufe des Root-Pfades der Webseite lassen darauf schließen, dass der Honeypot tatsächlich von circa 61 unterschiedlichen Systemen aufgerufen wurde. Auch die 14 Anfragen mit dem Ziel die Ressource 'favicon.ico' zu laden, lassen einen Rückschluss zu.  Bei dieser Ressource handelt es sich um ein Miniatur-Bild, ein Logo einer Webseite, das innerhalb eines Browser-Tabs der URI vorangestellt wird und im Wesentlichen der Repräsentation und Wiedererkennung dient. Das Anfordern dieser Ressource ist eine Eigenheit von verbreiteten Browsern, so lässt sich daraus schließen, dass es tatsächlich annähernd 14 manuell durchgeführte Aufrufe des Webhoneypots unter Verwendung eines Browsers gab. Da das Projektteam zu diesem Zeitpunkt bereits im Rahmen der Veröffentlichung dieser Dokumentation die IP-Adresse des Honeypots öffentlich gemacht hat, ist die Möglichkeit nicht auszuschließen, dass diese Zugriffe dieser Veröffentlichung zuzuschreiben sind und sonst möglicherweise nicht stattgefunden hätten.\\ 

Die wenig ansprechende Optik, der Mangel an Inhalten und Angriffsflächen in Form von Eingabefeldern der während dieses Zeitraums aktiven Weboberfläche des Honeypots, lassen darauf schließen, dass der Großteil der aufgezeichneten Zugriffe und Zugriffsversuche automatisiert stattgefunden hat. Dafür sprechen auch 5 Zugriffe auf die Datei 'robots.txt', die insbesondere von gutartigen aber auch bösartigen Bots, so genannten Webcrawlern, die das Internet automatisiert nach Inhalten durchsuchen, ausgewertet wird. So ist es eine gängige Praxis diese Datei nach Einträgen mit vorangestelltem 'disallow' zu durchsuchen, welche Pfade zu Ressourcen angibt, welche von einem solchen Webcrawler auf Wunsch des Administrators nicht betrachtet werden sollen. Gerade diese Vorgabe macht es für Angreifer interessant, genau diese, vom Administrator nicht zur Durchsicht erwünschten, Ressourcen aufzurufen und zu analysieren. 5 Zugriffsversuche auf die Webanwendung phpMyAdmin\footnote{ \textit{phpMyAdmin}: \url{https://www.phpmyadmin.net/}} lassen darauf schließen, 


Login-Formular nicht mehr relevant, nach Updates SNARE instabil.


5 phpmyadmin (MySQL verwaltung)Aufrufe -> Hätte sich gelohnt etwas entsprechendes zu simulieren
-> Botnetz
-> Bot sucht Schwachstellen in phpmyadmin -> https://ensourced.wordpress.com/2011/02/25/zmeu-attacks-some-basic-forensic/
http://stateofsecurity.com/?p=467
-> viele geskriptete Attacken auf phpmyadmin -> hohe Sicherheitsmaßnahmen nötig bei Vernwendung von phpmyadmin
-> http://sharadchhetri.com/2013/06/27/protect-from-w00tw00t-at-blackhats-romanian-anti-sec/

Insgesamt über 30 Versuche auf Login-Flächen zu gelangen oder Passwörter abzurufen -> Wichtigkeit sicherer Passwörter

HNAP  014, the HNAP was succesfully exploited by The Moon worm

http://stateofsecurity.com/?p=467 
   1 Request path: /nmaplowercheck1480791133
   1 Request path: /nice%20ports%2C/Tri%6Eity.txt%2ebak
   portscan mit nmap

w00tw00t -> This is just an automatic script deployed by many Script Kiddies looking for a security breach in your apache version/configuration. The signature w00tw00t is usually left by DFind.

./stssys.htm -> https://neonprimetime.blogspot.de/2016/08/stssyshtm-wget-request.html

Schlussfolgerung aufgrund von Aufmachung(zunächst example-Klon, danach schäbiges Login-Formular) -> Großteil der Angriffe automatisiert

-> Man hätte noch IP-Adressen auswerten können indem man python-file von SNARE editiert hätte -> Zeitgründe


Sample one:
