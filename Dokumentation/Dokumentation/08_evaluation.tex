\chapter{Evaluation}
\label{ch:Evaluation}

\section{Umsetzung der Anforderungen}
\label{sec:Umsetzung der Anforderungen}

Die unter \ref{sec:Muss-Kriterien} aufgeführten Muss-Kriterien wurden vollständig erfüllt. In diesem Projekt wurde ein Honeypotsystem auf einem von providerdienste.de bereitgestellten vServer umgesetzt. Dieser vServer ist über eine öffentliche IPv4-Adresse im Internet erreichbar und implementiert ein SSH- und Webhoneypot. Durch die von providerdienste.de bereitgestellte Konsole ist es möglich, das System selbst dann, wenn keine Verbindung via SSH möglich ist oder der Server kompromittiert oder gar übernommen wurde, zu konfigurieren und darauf zuzugreifen. Durch die Aktualisierung aller Pakete und der Limitierung der SSH-Authentifzierung auf ein Public-Key-Verfahren wurde das Honeypotsystem bestmöglich abgesichert. Jeglicher Datenverkehr wird geloggt.\\

Das Soll-Kriterium der automatischen Benachrichtigung bei einem Angriff auf das Honeypotsystem stellte sich im Nachhinein als nicht sonderlich hilfreich dar. Speziell auf Port 22 und somit auf den SSH-Honeypot wurden täglich viele Verbindungen aufgebaut. Die Benachrichtigung darüber war für die Projektmitglieder aber wenig hilfreich, da jegliche Kommunikation und Angriffsversuche geloggt und automatisiert ausgewertet wurden. Zudem werden die auswerteten Daten und Konfigurationen automatisch über eine verschlüsselte Verbindung auf dem Cloud-Speicher Google Drive abgelegt. Die Benachrichtigung bietet so nur einen geringen Mehrwert. Bei der Archivierung der auswerteten Daten sowie dem Backup von Konfigurationsdateien ist anzumerken, dass hier ein separater Server oder ein anderes Backupmedium möglicherweise besser geeignet wäre, da es mit den Richtlinien von Google zu einem Datenschutzkonflikt kommen kann.\\

Die Kann-Kriterien unter \ref{sec:Kann-Kriterien} wurden nur teilweise erfolgreich umgesetzt. So war es dem Projektteam aus technischen Gründen nicht möglich, neben dem Honeypotsystem weitere Systeme wie einen Router, eine Firewall oder PCs in die Infrastruktur zu integrieren. Auch die Umsetzung eines Honeypots, der ein offenes WLAN oder einen Fake-Acces-Point bietet konnte nicht umgesetzt werden. Hier bietet sich Verbesserungspotential, da die durch ein Honeypot gewonnen Informationen speziell in einer größeren Umgebung wertvoll sind. So ist denkbar, aus den gesammelten IP-Adressen Firwallregeln zu erstellen, die auf einer der Infrastruktur vorgelagerten Firewall zur Wirkung kommen.
Die komplette Umsetzung des Projekts fand auf einem gemieteten vServer statt. Die Kosten liegen hier mit 9,00 Euro je Monat in einem überschaubaren Rahmen. Die Anforderung nach einem geringen Stromverbrauch kann nicht exakt beurteilt werden, da über den Stromverbrauch keine Informationen vorliegen. Auf Grund dessen, dass es sich bei unserem System um ein virtuelles System handelt, welches sich ein physikalisches System mit anderen virtuellen Systemen teilt, kann jedoch von einer gewissen Energieeffizienz ausgegangen werden. Die Anforderung an einen reverse DNS-Lookup wurde wie unter x.y beschrieben umgesetzt und zusätzlich durch eine Darstellung einer Geo-IP-Karte (siehe Anhang x.y) ergänzt

\section{Schlussfolgerungen SSH-Honeypot}
\label{sec:Schlussfolgerungen SSH-Honeypot}

Wie der Statistik im Anhang unter \nameref{app:Pipal Passwortstatistik} entnommen werden kann, wurden am SSH-Honeypot-Port ingesamt 420 Passwörter eingegeben, 130 davon sind einzigartig. Die 10 meistgenutzten Passwörter machen 35,46\% aller Passworteingaben aus. Betrachtet man die 10 meistgenutzten Passwörter genauer, fällt auf, das hier vorrangig die Zeichenketten \textit{admin} und \textit{password} sowie Zahlenfolgen wie \textit{123456} vertreten sind. Die Länge der eingegeben Passwörter betrug in über 25\% der Fälle sechs Zeichen, in 15,71\% acht Zeichen und in 14,29\% vier Zeichen. Über 80\% der eingegeben Passwörter bestanden aus acht oder weniger Zeichen. Ähnlich klare Verteilungen sind bei der Zusammensetzung von Passwörter aus Zeichenklassen zu erkennen. So bestehen über 38\% der Passwörter rein aus Buchstaben, weitere 30,48\% rein aus Zahlen.\\

Eine ähnliche einseitige Verteilung lässt sich bei den verwendeten Benutzernamen erkennen. In über 50\% der Fälle wurde der Benutzername \textit{root} verwendet. Werden die Benutzernamen \textit{admin}, \textit{user} und \textit{test} hinzugezählt, decken diese Benutzernamen über 75\% ab.\\

Aus diesen Zahlen wird vor allem eines deutlich. Die Verwendung eines der oben genannten Passwörter in Verbindung mit einem der genannten Benutzernamen birgt ein erhebliches Sicherheitsrisiko. Werden die Zugriffszeiten der Angreifer auf den SSH-Honeypot berücksichtigt, kann bei der Verwendung dieser Benutzer-Passwort-Kombinationen von einer Kompromittierung innerhalb weniger Stunden oder gar Minuten ausgegangen werden. Es sollte so zwingend sichergestellt werden, dass ein sicheres Passwort verwendet wird. Besser ist die Authentifizierung rein auf das Public-Key-Verfahren zu beschränken.\\

Es muss angenommen werden, dass Angreifer nicht nur bei SSH-Zugriffen auf einfache Kombination aus Benutzernamen und Passwörter setzen. Jedoch ist bei anderen Diensten zur Passwortauthentifizierung oftmals keine Alternative vorgesehen. In diesem Fall sollte zwingend ein sicheres Passwort gewählt werden. Um die Definition eines sicheren Passworts zu veranschaulichen, leitet das Projektteam aus der Passwortstatistik eine Passwortrichtlinie ab. Diese sieht vor, dass ein sicheres Passwort folgende Eigenschaften besitzt:

\begin{itemize}
\item Passwortlänge von mindestens 10 oder mehr Zeichen
\item Kombination aus den Zeichenklassen Klein- und Großbuchstaben, Zahlen und Sonderzeichen
\item Benutzung von Wörtern und persönlichen Daten wie Namen und Geburtsdatum vermeiden
\end{itemize}

Da selbst im Bewusstsein dieser Richtlinien dazu tendiert wird, nicht sichere Passwörter zu genieren, empfiehlt das Projektteam die Verwendung eines Passwortmanagers wie KeePass\footnote{ \textit{KeePass}: \url{http://keepass.info/}}. KeePass ist für alle gängigen Betriebssysteme verfügbar und ermöglicht es rein zufällige Passwörter zu generieren, zu verwalten und sicher zu speichern. Somit kann auch der Empfehlung für jeden Dienst ein separates Passwort zu verwenden einfach nachgekommen werden.

\section{Schlussfolgerungen Web-Honeypot}
\label{sec:Schlussfolgerungen Web-Honeypot}

Aufgrund der späteren Inbetriebnahme des Web-Honeypots lassen sich zum jetzigen Zeitpunkt noch keine genaueren Rückschlüsse ziehen. Statistiken und Schlussfolgerungen basierend auf einer Auswertung geloggter Zugriffe werden nach ausreichender Laufzeit folgen.