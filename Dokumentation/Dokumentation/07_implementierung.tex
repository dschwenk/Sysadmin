\chapter{Implementierung}
\label{ch:Implementierung}


\section{Hostsystem}
\label{sec:Hostsystem}

- welches hostsystem mit welcher Konfiguration\\
- Benutzer für uns beide anlegen\\
- Konfiguation von Diensten wie SSH (Public Key Authentifizierung einrichten)\\
- beachten, dass Server-SSH-Dienst von Port 22 auf anderen Port gelegt wird (z.b 10022), da Honeypot-SSH-Dienst auf Port 22 laufen sollte\\
- vielelicht hinweis, dass über port scanner der normale ssh dienst trotzdem gefunden wird\\

\section{SSH-Honeypot}
\label{sec:SSH-Honeypot}

- vorgehenseise installation + konfiguration kippo
% http://www.blackhat.pm/ssh-honeypot-on-debian-with-kippo.html
% https://technik.blogbasis.net/kippo-ssh-honeypot-installieren-17-10-2014

- ausführliche Beschreibung Kippo + Dateien\\
% http://resources.infosecinstitute.com/tracking-attackers-honeypot-part-2-kippo/

- extract uniq ips aus kippo logfile \\
% https://bruteforce.gr/extracting-unique-ips-from-logfile.html \\

Die IP-Adressen von Angreifern werden im Kippo.log-Logfile neben zahlreichen Informationen wie bla %TODO
gespeichert. Ein Auszug aus dem Logfile ist im Anhang unter x zu finden.\\

Um aus den IP-Adressen Firewallregeln zu generieren (vgl. Kapitel x), müssen diese aus dem Logfile extrahiert werden. Dies geschieht mit Hilfe der Werkzeuge \textit{grep}, \textit{sort} und \textit{uniq}. Dazu wird an \textit{grep} ein regulären Ausdruck übergeben, der IP-Adressen filtert. Duplikate werden durch \textit{sort} und \textit{uniq} entfernt.

\begin{lstlisting}[style=customc]
cat logfile.log | 
  grep -o '[0-9]\{1,3\}\.[0-9]\{1,3\}\.[0-9]\{1,3\}\.[0-9]\{1,3\}' |
  sort |
  uniq > unique-ips.txt
\end{lstlisting}

Die Benutzernamen und Passwörter, die von Angreifern eingegeben werden, um sich zu authentifizieren werden ebenfalls extrahiert.

% http://bob.k6rtm.net/kippowho.html

\begin{lstlisting}[style=customc]
rep ' login attempt ' kippo.log |
  awk '{print ($9)}' |
  sort |
  uniq > user.txt
\end{lstlisting}

bla

\begin{lstlisting}[style=customc]
grep ' login attempt ' kippo.log |
  awk '{print ($9)}' |
  sed "s|^.*/||g" |
  sed "s|]||g" > pw.txt
\end{lstlisting}

Passwortduplikate werden hierbei nicht entfernt, um daraus aussagekräftige Statistiken generieren zu können.


- Auswertung der Passwörter (diese müssen zuerst geparst werden)\\
% https://github.com/digininja/pipal
% https://digi.ninja/blog/pipal_kippo.php
- alternativ Kippo Graph
% https://github.com/ikoniaris/kippo-graph

\section{Web-Honeypot}
\label{sec:Web-Honeypot}


\section{sonstiges}
\label{sec:sonstiges}

- passende überschriften finden\\
- archivieren von logfiles\\
- benachrichtigung bie angriff\\
- reverse lookup ip -adressen (+ darstellung auf karte? geo-ip?\\

