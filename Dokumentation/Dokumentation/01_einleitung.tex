\pagenumbering{arabic}
\chapter{Einleitung}
\label{ch:einleitung}

Das Internet und die Digitalisierung, die in alle Lebensbereiche Einzug hält, verändern Gesellschaft, Wirtschaft und Kultur. Egal ob im privaten oder beruflichen Umfeld, ständig sind wir von Computern in Form von Arbeitsgeräten, Smartphones oder anderen Geräten umgeben. Diese Verbreitung sowie die Vernetzung von Geräten untereinander wird in den nächsten Jahren im Zuge der "`Internet-der-Dinge-Evolution"' weiter drastisch zunehmen.\\


Ein oft vernachlässigter Aspekt ist hierbei das Thema "`IT-Sicherheit"'. Keine Software ist frei von Fehlern und Sicherheitslücken. Falsch konfigurierte Dienste und Software, die nicht regelmäßig aktualisiert wird, sind ein leichtes Ziel für Angreifer. Durch die zunehmende Vernetzung wird das Thema IT-Sicherheit in Zukunft weiter an Bedeutung gewinnen.\\

Um eine Infrastruktur, egal ob im privaten oder geschäftlichen Bereich, vor möglichen Angriffen zu schützen bedarf es eines immer größeren Aufwandes.



\section{Motivation}
\label{sec:Motivation}

Die Gewährleistung der IT-Sicherheit ist mittlerweile eine immens wichtige, wenn nicht sogar die wichtigste Anforderung an eine intakte IT-Infrastruktur. Entsprechend sollte ein Systemadministrator über ein breites Spektrum an Wissen im Bereich der IT-Sicherheit besitzen sowie in der Lage sein, mögliche Angriffsszenarien frühzeitig zu erkennen.\\


Das Konzept eines Honeypots, also einen potentiellen Angreifer nicht nur vor eigentlich wichtigen System fernzuhalten, sondern auch noch von seinem Wissen zu profitieren, stellt dabei einen hochspannenden Ansatz dar. 
Dieser Ansatz soll dem Projektteam helfen, Wissen über mögliche Angriffsszenarien und Vorgehensweisen zu erlangen, um so die Sicherheit von bestehenden und zukünftigen Infrastrukturen gewährleisten zu können.

\newpage


\section{Ziele}
\label{sec:Ziele}

Ziel dieser Arbeit ist es, ein System zu entwickeln, das als Honeypot dient. Dieser Honeypot soll eingesetzt werden, um Einblicke in die Vorgehensweise eines Angreifers zu bekommen. 
Das System stellt dazu ein vermeintlich leicht angreifbaren Webserver sowie SSH-Server dar.\\


Jegliche Zugriffe und Aktivitäten die ein Angriff hinterlässt werden protokolliert und ausgewertet. Das hierbei gewonnene Wissen soll in Form von IT-Sicherheitsmaßnahmen in bestehende und künftige IT-Infrastrukturen einfließen.

\begin{itemize}
\item Primärziel: einsatzfähige(r) Honeypot(s) - sicher, authentisch und lehrreich
\end{itemize}




\section{Eigene Leistung}
\label{sec:Eigene Leistung}

Der Hauptbestandteil dieses Projekts liegt in der Inbetriebnahme und Bereitstellung eines Honeypots, sowie die Integration desselben in eine für einen potentiellen Angreifer authentisch erscheinenden Umgebung. Dabei liegt das Hauptaugenmerk darauf, die Sicherheit des Systems zu gewährleisten. Die Dokumentation der Infrastruktur, der stattgefundener Angriffe sowie deren Auswertung stellen einen wichtigen Bestandteil dar. Dieses Wissen dient dem Projektteam zukünftig zur Absicherung von IT-Infrastruktur.

\begin{itemize}
\item Bereitstellung einer authentischen Umgebung für den Honeypot
\item Gewährleistung der Sicherheit für das eigene und das globale Netz
\item Inbetriebnahme des Honeypots selbst
\item automatisierte Auswertung von Log-Files durch Skripte
\item aus Auswertung erfolgt Erarbeitung und Ausweitung von Sicherheitsrichtlinien (Passwortrichtlinien, Firewallregeln)
\item Dokumentation von Honeypot inklusive Umgebung, Angriffsphase und Ergebnisse
\end{itemize}

