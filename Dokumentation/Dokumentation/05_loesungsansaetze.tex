\chapter{Lösungsansätze}
\label{ch:Lösungsansätze}

In den Anforderungen wurde definiert, dass das Honeypotsystem einen, oder besser mehrere Dienste anbieten soll. Um ein realistisches Angriffsziel abzugeben und die Dienste im Internet bereit zu stellen, wird ein Hostsystem mit öffentlicher IP-Adresse auf der Infrastruktur des Rechenzentrums aufgesetzt.

Aufgrund der begrenzten Ressourcen ist die Eigenentwicklung von Honeypotdiensten  nicht realisierbar. Ein Lösungsansatz für die Bereitstellung von einem SSH-Honeypot ist der Einsatz von Kippo. Damit wird ein SSH-Dienst, der nach erfolgreichem Login ein virtuelles System simuliert, bereitgestellt. Die von Kippo erzeugten Logfiles gilt es automatisiert auszuwerten. Dies wird mit Hilfe von einem Bash-Skript bewerkstelligt, dass IP-Adressen, Benutzernamen und Passwörter extrahiert. Aus den extrahierten IP-Adressen werden zyklisch Firewallregeln für iptables erzeugt, um zukünftige Angriffe von dieser IP zu blockieren.

Zur Bereitstellung von Diensten wie FTP oder HTTP wird ein separates Honeypot-Werkzeug eingesetzt, dass über eine Evaluation einer Anzahl an in Frage kommenden Systeme ausgewählt wird.
Um Logfiles und ausgewertete Daten zu archivieren, werden diese automatisch komprimiert und auf einem Cloud-Speicher wie GoogleDrive abgelegt. Eine Auswertung, aber auch Aufbereitung von Systemweiten Logfiles erfolgt mit Werkzeugen wie Logwatch oder Graylog.
Eine Benachrichtigung im Falle eines Zugriffs auf die Honeypotdienste kann mit dem Werkzeug “inotifywait” umgesetzt werden. Damit lassen sich Logfiles oder anderen Daten und Verzeichnisse auf Änderungen prüfen, um daraufhin eine Aktion wie den Versand einer Benachrichtigungsemail anzustoßen.

Um das System bestmöglich abzusichern werden nicht benötigte Dienste deaktiviert. Der Zugriff auf das Hostsystem für das Projektteam erfolgt via Public-Key-Authentifizierung über SSH. Dieser SSH-Dienst ist unabhängig vom Honeypot-SSH-Dienst.

Desweiteren bietet es sich an, das bestehende Honeypotsystem nach erfolgreicher Inbetriebnahme eines SSH-Honeypots um einen Web-Honeypot zu erweitern. Ein potentieller Lösungsansatz ist der Einsatz von Glastopf oder SNARE.
